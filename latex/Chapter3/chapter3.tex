%!TEX root = ../thesis.tex
%*******************************************************************************
%*********************************** Third Chapter *****************************
%*******************************************************************************

\chapter{Biological aspects of the epigenetic clock}  \label{c:3}

\ifpdf
    \graphicspath{{Chapter3/Figs/Raster/}{Chapter3/Figs/PDF/}{Chapter3/Figs/}}
\else
    \graphicspath{{Chapter3/Figs/Vector/}{Chapter3/Figs/}}
\fi


%********************************** %First Section  **************************************
\section{Background} 

Synchrony and asynchrony between an epigenetic clock and developmental timing
https://www.nature.com/articles/s41598-019-39919-3

\section{Discussion}

- Oscillatory amplitude is greatest at enhancers.  Maybe when DNMT3A is absent they are more sensitive to changes in DNA methylation, in this case towards hypomethylation. 
https://www.cell.com/cell-systems/fulltext/S2405-4712(18)30279-5

Add paragraph with discussion on epigenetic mitotic clock from TAC 2.


\section{Additional methods}

[For aDMPs we use linear regression, for Sotos DMPs, we use t-statistics]

In the case of a continuous phenotype (e.g. age) the association is carried out under a linear regression model framework, while for a binary phenotype (e.g. cancer/normal status) we use t-statistics [https://bmcbioinformatics.biomedcentral.com/articles/10.1186/1471-2105-13-59]

\subsection*{Experimental procedures for DNA methylation data generation}
\subsection*{Estimating epigenetic age acceleration in a biological system}

Include all the acceleration models (Horvath, pcgtAge, entropy)


