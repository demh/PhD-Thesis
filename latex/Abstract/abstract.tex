% ************************** Thesis Abstract *****************************
% Use `abstract' as an option in the document class to print only the titlepage and the abstract.
\begin{abstract}
Epigenetic clocks are mathematical models that predict the biological age of an organism using DNA methylation data, and which have emerged in the last few years as the most accurate biomarkers of the ageing process. However, little is known about the molecular mechanisms that control the rate of such clocks. In this thesis I focus on the study of the epigenetic ageing clock in humans. First, I review and benchmark statistical and computational tools required for the analysis of DNA methylation data in the context of human ageing. Next, I validate the performance of the Horvath epigenetic clock, the most widely used multi-tissue epigenetic clock in humans, in a control blood dataset and test its behaviour in patients with a variety of developmental disorders, which harbour mutations in proteins of the epigenetic machinery. I demonstrate that loss-of-function mutations in the H3K36 methyltransferase NSD1, which cause Sotos syndrome, substantially accelerate epigenetic ageing. Furthermore, I show that the normal ageing process and Sotos syndrome share methylation changes and the genomic context in which they happen. These results suggest that the H3K36 methylation machinery is a key component of the epigenetic maintenance system in humans, which controls the rate of epigenetic ageing, and this role seems to be conserved in model organisms. Finally, I provide a technological strategy to make epigenetic clocks (or any DNA methylation-based mathematical models) more cost-effective by exploiting the ability of restriction enzymes to perform genomic enrichment. This thesis provides novel insights (statistical, biological, technological) into the epigenetic ageing clock in humans, which will help to shed light on the different processes that erode the human epigenetic landscape during ageing.
\end{abstract}
